\section{Introducción}
El objetivo de este trabajo práctico es tratar de modelar, de la manera más fiel posible, un problema del mundo real mediante un \textbf{Diagrama Entidad-Relación}(DER). Luego de tener determinado mediante dicho diagrama, las \textit{entidades}, sus \textit{atributos} e \textit{interrelaciones} entre ellas, se pasará a definir el \textbf{Modelo Relacional}.\\
\\
\indent El Modelo Relacional derivado del DER es el que representa fielmente el futuro modelado físico en una base de datos real. Este consta de \textbf{tuplas} que representan las \textbf{tablas} de la base de datos física, junto a sus \textbf{columnas}.\\
\\
\indent Las columnas en el modelo físico tendrán los nombres aquí expuestos, habiendo además, dos tipos especiales: las \pk{primary key} (clave primaria de la tabla, subrayada con línea sólida) y las \fk{foreign key} (subrayada con línea punteada, hace referencia a la primary key de otra tabla).\\
\\
\indent La estructura del trabajo práctico seguirá el órden que tomamos para el desarrollo de las distintas partes del mismo, es decir, primero el DER, luego el Modelo Relacional derivado, y finalmente tenemos aparte la base de datos física junto con las consultas pedidas en el enunciado.

\subsection{Datos de ejemplo}
Hemos buscado datos de ejemplo para agregarle un poco de realismo a las tablas y, en consecuencia, a los resultados de las consultas sobre la base de datos, y hemos encontrado las siguientes fuentes de información:
\begin{itemize}
	\item \href{https://www.sat.gob.pe/websitev8/Modulos/contenidos/mult_Papeletas_ti_rnt.aspx}{\textbf{Servicio de Administración Tributaria de Lima}} para los detalles de infracciones.
	\item \href{http://www.buenosaires.gob.ar/areas/obr_publicas/lic_conducir/categorias.php?menu_id=6427}{\textbf{Categorías de licencias de conducir de la Ciudad de Buenos Aires}}.
\end{itemize}