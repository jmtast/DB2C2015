\section{Otras bases de datos NoSQL}

En esta \'ultima secci\'on del informe analizaremos la utilidad de otras bases de datos no relacionales, y en particular seleccionaremos una para hacer una comparaci\'on mas detallada sobre las limitaci\'ones y ventajas que presentar\'ia la implementaci\'on de las funcionalidades pedidas en el TP en otro tipo de bases de datos. Para comenzar, hay que aclarar que no todos los otros tipos de bases de datos NoSql estudiados en la materia son recomendables para la implementaci\'on de las funcionalidades pedidas en este trabajo. Esto tiene que ver con el aprobechamiento de las capacidades de la que dispone cada tipo de base de datos y con las limitaciones que presentan en la resoluci\'on de los conflictos encontrados. 

De esta forma, podemos citar el ejemplo de las Graph databases, con las cuales ser\'ia muy poco ventajosa la implementaci\'on del TP, ya que, por ejemplo en el primer punto, donde se pide mostrar una implementaci\'on desnormalizada del DER propuesto, los datos no est\'an organizados en forma de una red de entidades y relaciones, sino que tienen una representaci\'on mas natural como collecciones de entidades similares, lo cual es mas coherente con el modelo de document databases.

En el caso de las bases de datos orientadas a columnas, blah...

